\documentclass{article}
\usepackage[utf8]{inputenc}
\usepackage{hyperref}

\textheight=220mm
\textwidth=160mm
\voffset=-20mm
\hoffset=-12mm

\title{Mikalai Parafeniuk. Software developer.}
\date{\today}

\begin{document}

\maketitle

\section*{Contact Information}
\begin{itemize}
\item \textbf{Email} - physmatman@gmail.com
\item \textbf{Skype} - parmikola
\item \textbf{LinkedIn} - \url{https://www.linkedin.com/profile/view?id=177491081}
\item \textbf{GitHub} - \url{https://github.com/ParafeniukMikalaj}
\item \textbf{Phone} - +375 25 754 6119
\end{itemize}

\section*{Abstract}
I am a software developer with passion to challenging projects in BigData, HighLoad, Machine Learning, Algorithms, Distributed Systems. I have started programming when I was a freshman in university in 2009. Since then I went from simple Java tasks to Android, then to web applications and finally to complex distributed systems. In 2014 I received CS bachelors degree from Faculty of Computer Network and Systems of BSUIR and CS degree from Yandex school of data analysis. Currently I can't say, that I am SOME-TECHNOLOGY developer. I use any technology depending on project needs I am working with.

\section*{Work experience}

\begin{itemize}

\item MintiLab (since November 2014) \\\\
I moved to MintiLab because I was eager for development of BigData applications and wanted to go away from Android development. At this company buzz words like Kafka, Cassandra, Hadoop, Spark, RabbitMQ, ActiveMQ, ZooKeeper, Docker became not just buzz words, but technologies that are used every day. In this company you can't just say, that you are Java or Android developer and do only that kind of tasks. You should be a full stack developer, which knows how to write applications in different languages, can use cloud freely, can write client side code, can perform testing and perormance improvements of distributes systems.

The first project in this company was in field of financial markets. Lots of people want their money to work. The project our team worked for, provides people the ability to not just buy individual stocks, but create an investment profile. For example, person can invest to trend like Chinese Internet, and he will be advised a special set of stocks with percentage, to which money can be distributed. And the user want to be notified on some special events like change of account money, rising of account. So we solved the problem where lots of people with lots of investments profiles, which include lots of individual stocks, receive notifications they are interested in in a matter of seconds in reaction to rapid market changes.

The second project was aimed to improve transport system of San Francisco. A special service consists of premium class cars, which run on specified routes and can take passengers on and off. The challenge was to rewrite system completely (server, android/ios clients) and to handle amount of data generated by users.

The third and the most interesting project is a platform for processing of data streams. Platform started as an alternative for Apache Storm, which works as black box and don't give insights of what is going on, where the bottlenecks of the program exist. Our team proved that platform can be used for different kind of tasks: web crawling, ETL, IoT. Platform includes a big tool set to build applications, run and deploy them, control performance and execution process. The main aim of the platform is to shorten the gap between different types of users: business, data scientists, programmers, devops.

\item Yandex (June 2013 - November 2014) \\\\
I worked at Yandex as Android developer. Yandex has a lot of android programs: \\https://play.google.com/store/search?q=yandex. Most of them require authorization/registration module. I was responsible for support and development of that module. On this project I get an experience in solving compatibility problems, development of widely used library, providing good documentation for other application developers. I have also participated in development of alternative android store - Yandex.Store, and platform for unifying publishing process to different markets - One Platform Foundation.

\item Omegasoftware (January 2012 – June 2013) \\\\
I worked at OmegaSoftware as Web developer. I have started with support and development of web site for russian community of builders \url{http://nostroy.ru/}. I used Java Play Framework, ExtJS, MySQL during that project.

Later I have moved to another project - web site for corporate networks in Germany \url{http://www.channelplace.net}. Such technologies as Java, Asp.Net, Spring Framework, DevExpress, Javascript, jQuery, ExtJs, MySql, MS SQL, Windows Azure were used.

And the last project I participated at OmegaSoftware was a project for media company for improvement of advertising plan \url{http://mediacom.com}. Used Asp.Net and Oracle database on that project.
 
\item AndersenSoft (July 2011 – December 2011) \\\\
AndersenSoft was my first place to work as programmer. I started as Android developer. I worked on application for listening audiobooks on Android platform.

Application had an ability to listen to audiobooks with synchronized text, online audio books market, lots of authorization scenarios. App could be found here:\\https://play.google.com/store/apps/details?id=ru.librofon.
\end{itemize}

\section*{Education}
I received my bachelors degree in Computer Science from Belarussian State University of Informatics and Radioelectronics. I studied there from 2009 to 2014 at Faculty of Computer Networks and Systems on Informatics Department. University gave me strong math and programming background. At university I developed several interesting projects like cross-platform guitar tuner, rsa encryption utility, 1000 card game for Android, several sites, and lots of labs.\\\\
I have also studied in Yandex school of data analysis from 2012 to 2014 (\url{http://shad.yandex.ru/}). It improved my math and algorithms background and gave knowledge in such fields like machine learning, natural language processing, parallel programming. In shad I completed a lot of interesting homeworks: statistical translation utility, interesting machine learning competitions at \url{http://kaggle.com}, implementation of lamdas on C++03, named entity recognition, OpenMP, MPI, Wikipedia Page Rank on MapReduce and a lot of others.

\section*{Technical skills}

\begin{itemize}
\item \textbf{Professional interests}. Distributed systems, algorithms, machine learning, natural language processing, big data.
\item \textbf{Programming languages}. Primary languages: Java, C\#, Python. Secondary languages: C++, C, Javascript and used Delphi, Groovy, Scala and others some times.
\item \textbf{Frameworks and libraries}. I have used tons of libraries for Java: Netty, Apache Tomcat, Apache Zookeeper, Grizzly, Apache Hadoop, Apache Kafka, Apache Spark, Apache Cassandra. Spring, Play, Hibernate and others. In Android I used all possibilities of SDK, but have no experience in NDK usage. Also I used support library, ActionBar Sherlock, Volley and others. Among .NET libraries I can enumerate ASP.NET, ASP.NET MVC, Entity Framework, DevExpress.
\item \textbf{VCS}. Today I use Git actively also I had a lot of experience with SVN and I used Mercurial some times.
\item \textbf{IDE}. Android Studio, IntelliJ Idea, Eclipse, VisualStudio, Emacs, Vim.
\item \textbf{OS}. In different periods of my life I used most popular operating systems: MacOS, Linux, Windows.
\end{itemize}

\section*{Open source development}
After the completion of university and shad I am planning to contribute to open source development. As for now I have successfully participated in Google Summer Of Code students program in Apache organization in 2012. 

\end{document}
